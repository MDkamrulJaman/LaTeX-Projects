\documentclass{article}
\usepackage{graphicx}


\begin{center}


\includegraphics[width=0.5\textwidth]{Green_University_of_Bangladesh_logo.png} 


\textbf{\normalsize{ Green University of Bangladesh}} \\
\textbf{\normalsize{Department of Computer Science and Engineering (CSE)}}\\[1cm]

\textbf{\large Submission Date: 15/05/2023}
\vspace{1cm}

\textbf{\large Course Teacher's Name : Sagufta sabah nakshi}

\begin{center}
\textbf\large{Lab Report Proposal}

\end{center}

\date{\vspace{-5ex}}
\author{
    \begin{tabular}{cc}
        \multicolumn{2}{c}{\textbf{SUBMITTED TO:}} \\
        Name:Sagufta sabah nakshi       & \\
        Department of CSE & \\
        Green University of Bangladesh & \\
        \\
        \multicolumn{2}{c}{\textbf{SUBMITTED BY:}} \\
        Name:Mohammad Sajid Hossain  & Name:Md Salak Mahamud Pathan  \\
        ID:221902116  & ID:221902162  \\
        Department of CSE & Department of CSE \\
        Green University of Bangladesh & Green University of Bangladesh \\
    \end{tabular}
}

\date{\today}

\begin{document}
\newpage
\maketitle
\section*{TITLE OF THE Lab PROJECT PROPOSAL}
\begin{flushleft}
\textbf{Flip flop Led flasher}
\end{flushleft}

\section{PROBLEM DOMAIN \& MOTIVATIONS}
\subsection{PROBLEM DOMAIN}
A flip flop LED flasher is an electronic circuit that uses two flip-flops and a few other components, such as resistors, capacitors, and LED's, to create a flashing light effect. The circuit works by flipping between two states, causing an LED to turn on and off in a repeating pattern.

The two flip-flops are connected in what is known as a "toggle" configuration, where the output of one flip-flop is connected to the input of the other flip-flop, and vice versa. This creates a feedback loop that causes the circuit to oscillate between two states.

\subsection{MOTIVATIONS}

In one state, the output of one flip-flop is high, and the output of the other flip-flop is low. This causes one LED to turn on, while the other LED remains off. In the other state, the outputs of the flip-flops are reversed, causing the other LED to turn on and the first LED to turn off.
The timing of the circuit is controlled by the values of the resistors and capacitors used in the circuit. By adjusting these values, the frequency of the flashing effect can be changed.
Flip flop LED flashers are commonly used in decorative lighting, toys, and other electronic projects where a flashing light effect is desired.
\section{OBJECTIVES/AIMS}

The objectives or aims of a flip flop LED flasher circuit can vary depending on the specific application or project. Here are some common objectives:

\textbf{Visual Attention:} The primary aim of a flip flop LED flasher is to attract attention or create a visual effect. The flashing LEDs can be eye-catching and draw attention to a particular area or object.

\textbf{Signaling or Indication:} LED flashers are often used for signaling purposes. They can indicate the status of a device or system, such as indicating whether it's on or off, operational or in standby mode. The flashing LED's make it easier to notice and interpret the status information.
\textbf{Decorative Lighting:} LED flashers are popular for decorative lighting applications. They can be used in holiday decorations, party lighting, or artistic installations to create an interesting and dynamic lighting effect.
Educational or Learning Tool: Flip flop LED flasher circuits are often used in electronics education to teach basic principles of digital logic and sequential circuits. They provide a hands-on demonstration of how flip-flops can be used to create a simple sequential behavior.
\textbf{Prototyping and Experimentation:} LED flasher circuits can serve as a starting point for more complex electronic projects. They provide a simple platform for experimenting with timing, components, and circuit design. They can be modified and expanded upon to create more sophisticated flashing patterns or integrated into larger circuits.

\section{TOOLS \& TECHNOLOGIES}

\begin{itemize}
\item Breadboard 830 pt. - 1
\item Flip-flop ICs
\item Resistors (10K $\Omega$ - 2, 470 $\Omega$ - 2)
\item LED 5mm - 4
\item Jumper wires
\item Capacitor 100 $\mu$F - 2
\item Transistor BC547 - 2
\item Single strand wire 2m - 1
\item 9V Battery - 1
\item Battery snapper - 1
\end{itemize}
\section{CONCLUSION}
                          
In conclusion, a flip flop LED flasher is an electronic circuit that utilizes flip-flops, resistors, capacitors, LED's, and other components to create a flashing light effect. The objectives or aims of a flip flop LED flasher can vary, including attracting visual attention, signaling or indication, decorative lighting, educational purposes, and prototyping. The circuit can be built using tools and technologies such as a breadboard, flip-flop ICs, resistors, capacitors, LEDs, jumper wires, an oscilloscope for testing, and a soldering iron for permanent connections. By properly designing the circuit and adjusting the values of the components, a visually appealing and informative flashing effect can be achieved.



\end{document}